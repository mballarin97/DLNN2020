For the regression task it is better to use a reduced number of hidden layers, and in particular we choose to use $2$ of them.
Our network will so be composed of:
\begin{itemize}
    \item An input layer of one neuron;
    \item A first hidden layer oh $N_h$ neurons;
    \item A second hidden layer of $2\cdot N_h$ neurons;
    \item An output layer of one neuron.
\end{itemize}
To avoid vanishing problems in the gradient we will use a rectified linear unit (ReLU) as activation function.
This simple topology in the network is due to the fact that this problem is "simple" from the neural networks point of view, and adding too many parameters we 
will be easily led to overfitting. We will so make an extensive use of regularization procedures to avoid these problems. In 
particular, we will use two main methods:
\begin{enumerate}
    \item dropout, which consists of randomly eliminate neurons during the training with probability $p_{drop}$.    
    \item L2 regularization, which consists of adding a penalty in the loss proportional to the 2-norm
        of the parameters, i.e. weights and biases. We stress that in pytorch this method is enabled not in the loss
        function but in the optimizer, using the \lstinline{weight_decay} keyword.
\end{enumerate}
However, implementing both forms of regularizations generated sub-optimal results. We so decided to put to $0$ the L2
regularization, stressing that we can do this without losing generality.

We will try two different optimizers:
\begin{enumerate}
    \item The stochastic gradient descent with momentum, which is the really basic algorithm for optimization.
        The addition of the momentum, i.e. the memory of previous steps, really helps to go out of local minima and speeds up convergence;
    \item the Adam algorithm, adaptive moment estimator, a more advanced algorithm that uses estimations of first and second 
        moments  of the gradient to adapt the learning rate of each weight of the neural network.
\end{enumerate}
The amount of data available for this task is quite limited, we will so apply a cross-validation technique to perform the 
hyper parameters tuning. This method consists of splitting the training test into $k$ different folds and then, for each
hyper parameters set, train the network on the union of $k-1$ folds, performing the validation on the one which is missing.
This approach ensure also better performances, since the validation set is different for each of the iterations, and so we
cannot end up with a model that perfectly fits one given validation set.
After the hyper parameters choice is done, we will train the model over the full training test.

We will then optimize the number of hidden units in the layers of the network, the learning rate and the number of epochs
needed for the training procedure.

Lastly, we will visualize the network features: in particular we will plot the weights histograms and the activation profiles.
We will see that we can actually extract a lot of information, in particular from the activation profiles.